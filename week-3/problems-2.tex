\documentclass[a4paper]{scrartcl}

\usepackage{graphicx}
\usepackage[utf8]{inputenc}
\usepackage[bulgarian]{babel}
\usepackage{amsmath,amsfonts,amssymb,amsthm}
\usepackage{multicol}
\usepackage{pgf}
\usepackage{tikz}
\usetikzlibrary{arrows,automata}

\newcommand{\langdef}[5] { % {L}{w}{alphabet}{w math}{text} 
    \ensuremath{#1 = \left\{ #2 \in \{#3\}^* \mid #4 \right.} #5\ensuremath{\}.} 
}

\newcommand{\emptyw} {\epsilon}

\newcommand{\harder}{\emph{(по-трудна)}}

\newcommand{\state}[1]{ & #1 & }
\newcommand{\start}[1]{\to & #1 & }
\newcommand{\accep}[1]{ & #1 & {}^* }
\newcommand{\stacc}[1]{\to & #1 & {}^* }

\title{Задачки-закачки 2}
\subtitle{ЕАИ-упражнения, групи 2 и 3}

\begin{document}

\maketitle


\begin{enumerate} 

    \section*{детерминирани крайни автомати}
    
\item Проверете дали $A$ разпознава празната дума и потърсете по две други думи, които $A$ разпознава и по две думи, които $A$ не разпознава.
    Напишете регулярен израз за думите, които $A$ разпознава. Допълнете $A$ до тотален, ако не е.
    \begin{multicols}{2}
        \raggedright
        \begin{enumerate}
            \item \[\begin{array}{|r@{}c@{}l|c|c|c|}
                        \hline
                        \state{A} & 0 & 1 \\ \hline
                        \start{p} & q & r \\ \hline
                        \state{q} & r & p \\ \hline
                        \accep{r} & p & q \\ \hline
                \end{array}\]

            \item \[\begin{array}{|r@{}c@{}l|c|c|c|}
                        \hline
                        \state{A} & a & b \\ \hline
                        \stacc{s} & r & r \\ \hline
                        \state{q} & q & - \\ \hline
                        \state{r} & s & q \\ \hline
                \end{array}\]

            \item \[\begin{array}{|r@{}c@{}l|c|c|c|c|}
                        \hline
                        \state{A} & p & q & r \\ \hline
                        \accep{p} & r & r & - \\ \hline
                        \accep{q} & q & - & p \\ \hline
                        \start{r} & p & q & r \\ \hline
                \end{array}\]

            \item \[\begin{array}{|r@{}c@{}l|c|c|c|c|}
                        \hline
                        \state{A} & p & q & r \\ \hline
                        \accep{p} & - & - & - \\ \hline
                        \accep{q} & q & - & p \\ \hline
                        \start{r} & p & - & r \\ \hline
                \end{array}\]

            \item \[\begin{array}{|r@{}c@{}l|c|c|c|c|}
                        \hline
                        \state{A} & p & q & r \\ \hline
                        \accep{p} & - & - & - \\ \hline
                        \accep{q} & - & - & - \\ \hline
                        \start{r} & - & - & - \\ \hline
                \end{array}\]

            \item 
                \begin{tikzpicture}[>=stealth',shorten >=1pt,auto,node distance=2cm]
                    \node[initial,state] (s) {$s$};
                    \node[state] (p) [above right of=s] {$p$};
                    \node[state,accepting] (q) [below right of=p] {$q$};

                    \path[->] (s) edge node {$a,b$} (p);
                \end{tikzpicture}
        \end{enumerate}
    \end{multicols}

\end{enumerate}

\end{document}
